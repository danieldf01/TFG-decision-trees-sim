\apendice{Software Project Plan}

\section{Introduction}
This section presents how time management was handled in the course of this project. Throughout the development, bi-weekly meetings were held that served the purpose of reviewing what had been done and discussing what the tasks for the next sprint would be. To give better insight, a burndown of every sprint will be displayed. They are automatically created by Zube, the project agility tool that was used. The issues which these graphs are based on are all found in the ``Issues'' section\footnote{GitHub issues: \url{https://github.com/danieldf01/TFG-decision-trees-sim/issues}} of the project's GitHub repository.

\section{Time planning}
\subsection{Sprint 1 (29/02/2024 - 13/03/2024): Kick off project}
\paragraph{Objectives:}
The main objectives of this sprint were to set up the Github repository structure, link it to Zube for a better overview of each sprint's tasks, learn about decision trees and to create a first web application displaying a tree using SVG.

\paragraph{Results:}
Almost all the tasks that were intended for this sprint were completed, except for the documentation of the Decision Trees concept in the Memoria.

Figure \ref{fig:sprint1_burndown} shows the burndown of the sprint.
\imagen{sprint1_burndown}{Burndown Sprint 1}

\subsection{Sprint 2 (14/03/2024 - 03/04/2024): Implementation of tree graphics}
\paragraph{Objectives:}
For this sprint, the intention was to create the first two prototypes, one displaying the entropy function with a calculator and the other one displaying a decision tree, both making use of the D3.js library. To display these prototypes, a GitHub Pages repository was to be created. Solidifying knowledge about conditional entropy and making entries to the "Theoretical concepts" section of the Memoria were also part of this sprint. 

\paragraph{Results:}
As seen on the burndown in figure \ref{fig:sprint2_burndown}, everything was completed except for the prototype displaying a decision tree. Due to sickness during the sprint, this task was left unfinished and pushed back to a later sprint for the time being.
\imagen{sprint2_burndown}{Burndown Sprint 2}

\subsection{Sprint 3 (04/04/2024 - 17/04/2024): Prototype for conditional Entropy}
\paragraph{Objectives:}
During this sprint, the main tasks were to refactor the GitHub repository structure, upgrade the visual presentation of the Entropy prototype using the Bootstrap framework, start documenting technical tools used in the Memoria and to create a prototype displaying a calculator for conditional Entropy.

\paragraph{Results:}
As figure \ref{fig:sprint3_burndown} shows, all the tasks of this sprint were completed in time.
\imagen{sprint3_burndown}{Burndown Sprint 3}

\subsection{Sprint 4 (18/04/2024 - 02/05/2024): Prototype Decision Tree} \label{sprint_4}
\paragraph{Objectives:}
The main tasks of this sprint were to, on one hand, improve the existing prototypes with exceptions and enhance the overall code quality and, on the other hand, create a prototype that displays a decision tree based on an example dataset. Besides that, it was also asked to continue working on the Memoria by documenting some technical environments that were used.

\paragraph{Results:}
As seen in figure \ref{fig:sprint4_burndown}, all tasks were completed except for two issues regarding the documentation of related works and a theoretical concept. This shortcoming was due to time constraints caused by assignments and exams in other classes.
\imagen{sprint4_burndown}{Burndown Sprint 4}

\subsection{Sprint 5 (03/05/2025 - 16/05/2024): step-by-step Decision Tree simulation}
\paragraph{Objectives:}
This sprint's main objective consisted of implementing a step-by-step visualization for the decision tree prototype that was created in the previous sprint. To achieve that, the decision tree creation had to be made dynamic, which, at the time, it was not. Other tasks included the creation of a header and footer for the web application and documenting relevant aspects of the development.

\paragraph{Results:}
Figure \ref{fig:sprint5_burndown} displays this sprint's burndown which shows that all the proposed tasks were done in time.
\imagen{sprint5_burndown}{Burndown Sprint 5}

\subsection{Sprint 6 (17/05/2024 - 30/05/2024): Decision Tree value table, CSV data loading, interactive data}
\paragraph{Objectives:}
One of this sprint's main goals was to upgrade the decision tree's prototype by adding a dynamic value table that would display relevant values, like each feature's information gain, at each step. The other main objectives were to make it possible for the user to use their own datasets in CSV file format and to allow them to add and remove rows and columns from a currently loaded dataset.

\paragraph{Results:}
Figure \ref{fig:sprint6_burndown} shows that, due to the sprint having been during the final exam phase, not all tasks were completed. Besides issues like scaling text sizes based on their width and a cleanup of the project layout, one of the main objectives was left unfinished. While the addition of user-uploaded CSV datasets was successful, the ``interactive data'' goal was not met. In the end, it was discarded altogether as other refinements took priority due to the lack of time.
\imagen{sprint6_burndown}{Burndown Sprint 6}

\subsection{Sprint 7 (31/05/2024 - 06/06/2024): Decision Tree selectable example data, CSV file requirements}
\paragraph{Objectives:}
The final sprint of this project's development was used to refine some of the already existing parts of the application. One issue was to add the functionality of being able to choose between different example datasets for the decision tree ID3 simulation. Another was to formulate requirements that a user-chosen CSV dataset had to meet and display them.

\paragraph{Results:}
Figure \ref{fig:sprint7_burndown} shows a burndown of the final sprint. As this sprint still took place during the exam phase, not all tasks could be finished here either.  However, those were only minor issues like an improvement of the repository's README file which could were completed in the final days before the deadline.
\imagen{sprint7_burndown}{Burndown Sprint 7}

\subsection{Overview}
Figure \ref{fig:sprints_overview} displays an overview of the amount of issues that were resolved during each sprint.
\imagen{sprints_overview}{Sprints overview}
The workload was visibly increased starting at sprint 4 (\ref{sprint_4}), which marked the start of the implementation of the decision tree prototype. Before that point, a lot of the sprints' work included learning and getting used to new technologies in order to be able to use them. Most of this process was not captured in the form of issues. 

\section{Viability study}

\subsection{Economic viability}
\subsubsection{Staff costs}
This project has been carried out by one programmer over the course of approximately 3 months. Taking into consideration the developer's status of full-time student who also had to spend time in preparation for other classes, the overall development time can be estimated to a part-time employment. Considering the monthly minimum wage for general workers in Spain in 2024 of 1134,00€ per month~\cite{minimum_wage_spain_2024}, the estimated salary would be:

\[ 3m * 1134\text{€}/m = 3402\text{€} \]

As this represents only the gross wage, social security contributions have to be added, as well. As of 2024, these make up 36,85\%, with the employee having to contribute 6,45\% and the employer 30,4\%~\cite{spain_taxes}. Adding this to the gross wage totals up to:

\[ 3402\text{€} + 3402\text{€} * 0,3685 = 4655,637\text{€} \]

\subsubsection{Software and hardware costs}
The software cost of this project is equal to 0 as only free software has been used. As for hardware, a device with a value of 863,27€ has been used to carry out this project. Depreciation does not have to be considered as the device was bought at the beginning of development. Having used this device for the project over the course of 3 months, it makes up for a total cost of:

\[ 3m * 863,27\text{€}/m = 2589,81\text{€} \]

\subsubsection{Total cost}
Considering staff and hardware costs together, the total cost is summed up to:

\[ 2589,81\text{€} + 4655,637\text{€} = 7245,447\text{€} \]
\pagebreak

\subsection{Legal viability}
Table \ref{tabla:table_dependencies} shows every used dependency, its version and license

\tablaSmallSinColores{Every dependency used in the project and its license}{l c c}{table_dependencies}
{ \multicolumn{1}{l}{Dependency} & Version & License \\}{ 
jQuery & 3.7.1 & MIT \\
jest & 29.7.0 & MIT \\
jest-environment-jsdom & 29.7.0 & MIT \\
D3 & 7.9.0 & ISC \\
Bootstrap & 5.3.3 & MIT \\
bootstrap-icons & 1.11.3 & MIT \\
MathJax & 3.2.2 & Apache-2.0 \\
Polyfill service & 3.25.3 & CC0-1.0 \\
@popperjs/core & 2.11.8 & MIT \\
papaparse & 5.4.1 & MIT \\
} 

The most restrictive of these licenses would be the Apache-2.0 license, which is still a permissive license that allows, e.g., the software being used, modified and distributed in a commercial context.
