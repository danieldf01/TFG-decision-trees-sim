\apendice{Plan de Proyecto Software}

\section{Introducción}

\section{Planificación temporal}
\subsection{Sprint 1 (29/02/2024 - 13/03/2024): Kick off project}
\paragraph{Objectives:}
The main objectives of this sprint were to set up the Github repository structure, link it to Zube for a better overview of each sprint's tasks, learn about decision trees and to create a first web application displaying a tree using SVG.

\paragraph{Results:}
Almost all the tasks that were intended for this sprint were completed, except for the documentation of the Decision Trees concept in the Memoria.

You can see the burndown of the sprint in the following graph.

\imagen{sprint1_burndown}{Burndown Sprint 1}

\subsection{Sprint 2 (14/03/2024 - 03/04/2024): Implementation of tree graphics}
\paragraph{Objectives:}
For this sprint, the intention was to create the first two prototypes, one displaying the entropy function with a calculator and the other one displaying a decision tree, both making use of the D3.js library. To display these prototypes, a Github page was to be created. Solidifying knowledge about conditional entropy and making entries to the "theoretical concepts" section of the Memoria were also part of this sprint. 

\paragraph{Results:}
As seen on the burndown in the following image, everything was completed except for the prototype displaying a decision tree. Due to sickness during the sprint, this task was left unfinished.

\imagen{sprint2_burndown}{Burndown Sprint 2}

\subsection{Sprint 3 (04/04/2024 - 17/04/2024): Prototype for conditional Entropy}
\paragraph{Objectives:}
During this sprint, the main tasks were to refactor the Github repository structure, upgrade the visual presentation of the Entropy prototype using the Bootstrap framework, start documenting technical tools used in the Memoria and to create a prototype displaying a calculator for conditional Entropy.

The task of creating a prototype displaying a decision tree was pushed back for this sprint.

\paragraph{Results:}
As you can see on the burndown in the following image, all the tasks of this sprint have been completed in time.

\imagen{sprint3_burndown}{Burndown Sprint 3}

\subsection{Sprint 4 (18/04/2024 - 02/05/2024): Prototype Decision Tree}
\paragraph{Objectives:}
The main tasks of this sprint were to, on one hand, improve the existing prototypes with exceptions and enhance the overall code quality and, on the other hand, create a prototype that displays a decision tree based on an example dataset. Besides that, it was also asked to continue working on the Memoria by documenting some technical environments that were used.

\paragraph{Results:}
As seen in the following graph, except for two issues regarding the documentation of related works and a theoretical concept, were completed. This, was due to time constraints caused by assignments and exams in other classes.

\imagen{sprint4_burndown}{Burndown Sprint 4}

\subsection{Sprint 5 (03/05/2025 - 16/06/2024): step-by-step Decision Tree simulation}
\paragraph{Objectives:}
This sprint's main objective consisted of implementing a step-by-step visualization for the decision tree prototype that was created in the previous sprint. To achieve that, the decision tree creation had to be made dynamic, which, at the time, it was not. Other tasks included the creation of a header and footer for the web application and documenting relevant aspects of the development.

\paragraph{Results:}
The following graph shows that all the proposed tasks were able to be completed in time.

\imagen{sprint5_burndown}{Burndown Sprint 5}

\section{Estudio de viabilidad}

\subsection{Viabilidad económica}

\subsection{Viabilidad legal}