\capitulo{1}{Introduction}

Decision trees belong to the most popular models in machine learning and data mining, serving as intuitive models for decision-making. The algorithm known as ID3, or Iterative Dichotomiser 3, recursively partitions data based on attribute values with the aim of maximizing information gain at each step. It is one of the most popular decision tree algorithms and has significantly contributed to their widespread use.

This project has drawn great inspiration from the ``Seshat Tool''\footnote{Seshat Tool: \url{http://cgosorio.es/Seshat/}} and the research article~\cite{https://doi.org/10.1002/cae.22036} based on it. Its aim is to facilitate users' learning experiences about a concept through intuitive step-by-step simulations of certain algorithms. In the case of Seshat Tool, the target concepts were lexical analysis algorithms. While the main influences are its layout and overall functionality of the presented web application, the related article also carried out a study on its actual effectiveness.

Figure \ref{fig:seshat_student_tests} shows a table out of the article that presents results of a questionnaire that students were asked to do. They took the questionnaire after they ``were taught the main basic concepts of automata and regular expressions, as well as''~\cite{https://doi.org/10.1002/cae.22036} two algorithms that are also taught in the web application. The students were split into two groups, one that would take the test only based on the knowledge gained by a prior theoretical lecture and another that would additionally be granted access to the Seshat Tool.
\imagen{seshat_student_tests}{Results of students' tests}{1}
It can be observed that the group that had access to the tool achieved noticeably better results overall than the other group.

Following the success of the Seshat Tool, this project's aim has been to create a similarly helpful and easy-to-understand decision tree simulator in form of a web application that teaches the ID3 algorithm and all necessary surrounding concepts.