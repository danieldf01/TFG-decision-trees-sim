\apendice{Requirements Specification}

\section{Introduction}
This section will explain the requirements of the application by specifying (non-)functional requirements and use cases.

\section{General objectives}
The main objective of this project has been to create a web application under the name of "Decision Tree Simulator" and with the purpose of helping users learn the concept of decision trees, how they are created and all necessary surrounding topics in an intuitive and simple way.

It provides dynamic calculators for entropy and conditional entropy which let the user input values to they can observe how different values affect the results. There is also a visual representation of the binary entropy graph that uses SVG and responds with markers to the user's input, if they used two classes to calculate the entropy.

The decision tree simulation presents a step-by-step visualization of the ID3 algorithm so that each user can follow the steps at their own pace. They can choose between selecting one of the example datasets or loading their own dataset in CSV file format. With a combination of a decision tree that is dynamically created using SVG, a dataset table, a value table, and visual cues at each step, the goal was to make the user's learning experience simple and intuitive.

\section{Requirements catalog}
\subsection{Functional Requirements}
\begin{itemize}
    \item \textbf{FR-1} From the web, it must be possible to run the Entropy calculator for calculating the entropy of given input values
    \begin{itemize}
        \item \textbf{FR-1.1} The user must be able to enter values into the presented input fields which are positioned in the column that is given the name ``Nr. of instances'' by the respective column header.
        \item \textbf{FR-1.2} The user must be able to add classes by clicking on the button labeled ``+''.
        \item \textbf{FR-1.3} The user must be able to remove the row that represents the class that was last added by clicking on the button labeled ``-''.
        \item \textbf{FR-1.4} The user must be able to initialize the calculation of the entropy by clicking on the button labeled ``Calculate Entropy''.
        \item \textbf{FR-1.5} The application must, given valid input values, correctly calculate each class's p-value and the feature's entropy and display those values on the corresponding Entropy table.
        \item \textbf{FR-1.6} The application must, if only 2 classes were used, show the results of the entropy calculation through a red dot on the x-axis of the presented coordinate system and a red line pointing to the corresponding point on the presented Binary Entropy graph.
    \end{itemize}
\end{itemize}

\begin{itemize}
    \item \textbf{FR-2} From the web, it must be possible to run the Conditional Entropy calculator for calculating the conditional entropy of given input values.
    \begin{itemize}
        \item \textbf{FR-2.1} The user must be able to enter values into the presented input fields which are positioned in the columns that are given the name ``Class 1" and ``Class 2" by the respective column headers.
        \item \textbf{FR-2.2} The user must be able to add categories by clicking on the button labeled ``+''.
        \item \textbf{FR-2.3} The user must be able to remove the row that represents the category that was last added by clicking on the button labeled ``-''.
        \item \textbf{FR-2.4} The user must be able to initialize the calculation of the conditional entropy by clicking on the button labeled ``Calculate Conditional Entropy''.
        \item \textbf{FR-2.5} The application must, given valid input values, correctly calculate each category's ratio, entropy, and the feature's conditional entropy and display those values on the table.
    \end{itemize}
\end{itemize}

\begin{itemize}
    \item \textbf{FR-3} From the web, it must be possible to run the Decision Tree simulator for executing a step-by-step simulation of the ID3 algorithm.
    \begin{itemize}
        \item \textbf{FR-3.1} The user must be able to choose a dataset from one of the example datasets that are provided by the web application.
        \item \textbf{FR-3.2} The user must be able to select their own dataset in a CSV file format.
        \item \textbf{FR-3.3} The application must, given a valid CSV file, load the dataset that is contained in the file.
        \item \textbf{FR-3.4} The application must, following a successful load of a dataset, display an information card in regards to the chosen dataset, the root node of the dynamically created decision tree, a data table presenting the dataset, and a value table that presents values that are relevant to the decision tree's creation at each step.
        \item \textbf{FR-3.5} The user must be able to navigate through the step-by-step simulation with the use of the four buttons that represent the four functions ``Initial step'', ``Step back'', ``Step forward'', and ``Last step'', respectively.
        \item \textbf{FR-3.6} The application must, given that the ``Initial step'' button was clicked by the user, go to the first step of the simulation.
        \item \textbf{FR-3.7} The application must, given that the ``Step back'' button was clicked by the user and the simulation had not already been at the first step, go back one step in the simulation.
        \item \textbf{FR-3.8} The application must, given that the ``Step forward'' button was clicked by the user and the simulation had not already been at the last step, go forward one step in the simulation.
        \item \textbf{FR-3.9} The application must, given that the ``Last step'' button was clicked by the user, go to the last step of the simulation.
    \end{itemize}
\end{itemize}

\subsection{Non-functional requirements}
\begin{itemize}
    \item \textbf{NFR-1} The user interface must be simple and intuitive.
    \item \textbf{NFR-2} The application must be responsive to different screen sizes.
    \item \textbf{NFR-3} For the Entropy calculator and Conditional Entropy calculator, the application must recognize any positive integer value as valid input.
    \begin{itemize}
        \item \textbf{NFR-3.1} The user must be warned through appearing alerts if any of the user-made inputs is invalid.
    \end{itemize}
    \item \textbf{NFR-4} For the Entropy calculator, if more than 2 classes are used, the user must be informed through an appearing alert about the fact that the calculated results will not be displayed on the Binary Entropy graph.
    \item \textbf{NFR-5} For the Decision Tree simulator, the application must recognize CSV files that meet the file requirements that are displayed in the application as valid.
    \begin{itemize}
        \item \textbf{NFR-5.1} The user must be warned through an appearing alert if the proposed CSV file fails to meet any of the requirements and is therefore recognized as invalid.
    \end{itemize}
\end{itemize}

\section{Requirements specification}
\subsection{Use case diagram}

\subsection{Use cases}
% Use case 1
\begin{table}[p]
	\centering
	\begin{tabularx}{\linewidth}{ p{0.21\columnwidth} p{0.71\columnwidth} }
		\toprule
		\textbf{UC-1}    & \textbf{Run Entropy calculator}\\
		\toprule
		\textbf{Version}              & 1.0    \\
		\textbf{Author}                & Daniel Drefs Fernandes \\
		\textbf{Associated requirements} & FR-1, FR-1.1, FR-1.2, FR-1.3, FR-1.4, FR-1.5, FR-1.6 \\
		\textbf{Description}          & The user runs the Entropy calculator with the desired input values and receives the results in a visual format on the website. \\
		\textbf{Precondition}         & The input values introduced by the user are valid. \\
		\textbf{Actions}             &
		\begin{enumerate}
			\def\labelenumi{\arabic{enumi}.}
			\tightlist
			\item The user opens the application.
            \begin{enumerate}
                \item The user adds one or multiple class by clicking the button with the label ``+''.
            \end{enumerate}
			\item The user fills the input fields with the desired values and clicks on the button with the label ``Calculate Entropy''.
            \item The application calculates each class's p-value and the feature's entropy and displays the results on the corresponding table.
            \begin{enumerate}
                \item If the user has not added any classes, the application will show a visualization of the calculated results in SVG format on the Binary Entropy graph.
            \end{enumerate}
		\end{enumerate}\\
		\textbf{Postcondition}        & The results are displayed on the Entropy table. \\
		\textbf{Exceptions}          & If the user has introduced invalid values, the application will display an alert and inform the user to only use positive integer values. \\
		\textbf{Importance}          & High \\
		\bottomrule
	\end{tabularx}
	\caption{UC-1 Run Entropy calculator.}
\end{table}

\subsection{Use cases}
% Use case 2
\begin{table}[p]
	\centering
	\begin{tabularx}{\linewidth}{ p{0.21\columnwidth} p{0.71\columnwidth} }
		\toprule
		\textbf{UC-2}    & \textbf{Run Conditional Entropy calculator}\\
		\toprule
		\textbf{Version}              & 1.0    \\
		\textbf{Author}                & Daniel Drefs Fernandes \\
		\textbf{Associated requirements} & FR-2, FR-2.1, FR-2.2, FR-2.3, FR-2.4, FR-2.5 \\
		\textbf{Description}          & The user runs the Conditional Entropy calculator with the desired input values and receives the results in a visual format on the website. \\
		\textbf{Precondition}         & The input values introduced by the user are valid. \\
		\textbf{Actions}             &
		\begin{enumerate}
			\def\labelenumi{\arabic{enumi}.}
			\tightlist
			\item The user opens the application.
            \begin{enumerate}
                \item The user adds one or multiple categories by clicking the button with the label ``+''.
            \end{enumerate}
			\item The user fills the input fields with the desired values and clicks on the button with the label ``Calculate Conditional Entropy''.
            \item The application calculates each category's ratio, entropy, and the feature's conditional entropy and displays the results on the table.
		\end{enumerate}\\
		\textbf{Postcondition}        & The results are displayed on the table. \\
		\textbf{Exceptions}          & If the user has introduced invalid values, the application will display an alert and inform the user to only use positive integer values. \\
		\textbf{Importance}          & High \\
		\bottomrule
	\end{tabularx}
	\caption{UC-2 Run Conditional Entropy calculator.}
\end{table}

\subsection{Use cases}
% Use case 3
\begin{table}[p]
	\centering
	\begin{tabularx}{\linewidth}{ p{0.21\columnwidth} p{0.71\columnwidth} }
		\toprule
		\textbf{UC-3}    & \textbf{Run Decision Tree simulator}\\
		\toprule
		\textbf{Version}              & 1.0    \\
		\textbf{Author}                & Daniel Drefs Fernandes \\
		\textbf{Associated requirements} & FR-3, FR-3.1, FR-3.2, FR-3.3, FR-3.4, FR-3.5, FR-3.6, FR-3.7, FR-3.8, FR-3.9 \\
		\textbf{Description}          & The user runs the Decision Tree simulator with the desired dataset, receives the results in a visual format on the website and goes through the step-by-step simulation. \\
		\textbf{Precondition}         & The CSV file introduced by the user is valid. \\
		\textbf{Actions}             &
		\begin{enumerate}
			\def\labelenumi{\arabic{enumi}.}
			\tightlist
			\item The user opens the application.
			\item The user chooses a dataset from one of the example datasets that are provided by the application.
            \begin{enumerate}
                \item Alternatively, the user chooses their own dataset in a CSV file format.
            \end{enumerate}
            \item The application loads the dataset and displays an information card designated for the dataset, the root node of the decision tree, a data table corresponding to the dataset, and the value table.
            \item The user uses the four presented buttons to navigate through the step-by-step simulation.
		\end{enumerate}\\
		\textbf{Postcondition}        & The decision tree, data table, and value table are displayed at the user's desired step of the simulation. \\
		\textbf{Exceptions}          & If the user has introduced an invalid CSV file, the application will display an alert and inform the user to check the file requirements. \\
		\textbf{Importance}          & High \\
		\bottomrule
	\end{tabularx}
	\caption{UC-3 Run Decision Tree simulator.}
\end{table}
