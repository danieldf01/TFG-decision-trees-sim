\apendice{Requirements Specification}

\section{Introduction}

Una muestra de cómo podría ser una tabla de casos de uso:

% Caso de Uso 1 -> Consultar Experimentos.
\begin{table}[p]
	\centering
	\begin{tabularx}{\linewidth}{ p{0.21\columnwidth} p{0.71\columnwidth} }
		\toprule
		\textbf{CU-1}    & \textbf{Ejemplo de caso de uso}\\
		\toprule
		\textbf{Versión}              & 1.0    \\
		\textbf{Autor}                & Alumno \\
		\textbf{Requisitos asociados} & RF-xx, RF-xx \\
		\textbf{Descripción}          & La descripción del CU \\
		\textbf{Precondición}         & Precondiciones (podría haber más de una) \\
		\textbf{Acciones}             &
		\begin{enumerate}
			\def\labelenumi{\arabic{enumi}.}
			\tightlist
			\item Pasos del CU
			\item Pasos del CU (añadir tantos como sean necesarios)
		\end{enumerate}\\
		\textbf{Postcondición}        & Postcondiciones (podría haber más de una) \\
		\textbf{Excepciones}          & Excepciones \\
		\textbf{Importancia}          & Alta o Media o Baja... \\
		\bottomrule
	\end{tabularx}
	\caption{CU-1 Nombre del caso de uso.}
\end{table}

\section{General objectives}

\section{Requirements catalog}
\subsection{Functional Requirements}
\begin{itemize}
    \item \textbf{FR-1} From the web, it must be possible to run the Entropy calculator for calculating the entropy of given input values
    \begin{itemize}
        \item \textbf{FR-1.1} The user must be able to enter values into the presented input fields which are positioned in the column that is given the name ``Nr. of instances'' by the respective column header.
        \item \textbf{FR-1.2} The user must be able to add classes by clicking on the button labeled ``+''.
        \item \textbf{FR-1.3} The user must be able to remove the row that represents the class that was last added by clicking on the button labeled ``-''.
        \item \textbf{FR-1.4} The user must be able to initialize the calculation of the entropy by clicking on the button labeled ``Calculate Entropy''.
        \item \textbf{FR-1.5} The application must, given valid input values, correctly calculate each class's p-value and the feature's entropy and display those values on the right table.
        \item \textbf{FR-1.6} The application must, if only 2 classes were used, show the results of the entropy calculation through a red dot on the x-axis of the presented coordinate system and a red line pointing to the corresponding point on the presented Binary Entropy graph.
    \end{itemize}
\end{itemize}

\begin{itemize}
    \item \textbf{FR-2} From the web, it must be possible to run the Conditional Entropy calculator for calculating the conditional entropy of given input values.
    \begin{itemize}
        \item \textbf{FR-2.1} The user must be able to enter values into the presented input fields which are positioned in the columns that are given the name ``Class 1" and ``Class 2" by the respective column headers.
        \item \textbf{FR-2.2} The user must be able to add categories by clicking on the button labeled ``+''.
        \item \textbf{FR-2.3} The user must be able to remove the row that represents the category that was last added by clicking on the button labeled ``-''.
        \item \textbf{FR-2.4} The user must be able to initialize the calculation of the conditional entropy by clicking on the button labeled ``Calculate Conditional Entropy''.
        \item \textbf{FR-2.5} The application must, given valid input values, correctly calculate each category's ratio, entropy, and the feature's conditional entropy and display those values on the table.
    \end{itemize}
\end{itemize}

\subsection{Non-functional requirements}
\begin{itemize}
    \item \textbf{NFR-1} The user interface must be simple and intuitive.
    \item \textbf{NFR-2} The application must be responsive to different screen sizes.
    \item \textbf{NFR-3} For the Entropy calculator and Conditional Entropy calculator, the application must recognize any positive integer value as valid input.
    \begin{itemize}
        \item \textbf{NFR-3.1} The user must be warned through appearing alerts if any of the user-made inputs is invalid.
    \end{itemize}
    \item \textbf{NFR-4} For the Entropy calculator, if more than 2 classes are used, the user must be informed through an appearing alert about the fact that the calculated results will not be displayed on the Binary Entropy graph.
\end{itemize}

\section{Requirements specification}