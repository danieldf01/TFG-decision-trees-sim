\apendice{Requirements Specification}

\section{Introduction}

Una muestra de cómo podría ser una tabla de casos de uso:

\section{General objectives}

\section{Requirements catalog}
\subsection{Functional Requirements}
\begin{itemize}
    \item \textbf{FR-1} From the web, it must be possible to run the Entropy calculator for calculating the entropy of given input values
    \begin{itemize}
        \item \textbf{FR-1.1} The user must be able to enter values into the presented input fields which are positioned in the column that is given the name ``Nr. of instances'' by the respective column header.
        \item \textbf{FR-1.2} The user must be able to add classes by clicking on the button labeled ``+''.
        \item \textbf{FR-1.3} The user must be able to remove the row that represents the class that was last added by clicking on the button labeled ``-''.
        \item \textbf{FR-1.4} The user must be able to initialize the calculation of the entropy by clicking on the button labeled ``Calculate Entropy''.
        \item \textbf{FR-1.5} The application must, given valid input values, correctly calculate each class's p-value and the feature's entropy and display those values on the corresponding Entropy table.
        \item \textbf{FR-1.6} The application must, if only 2 classes were used, show the results of the entropy calculation through a red dot on the x-axis of the presented coordinate system and a red line pointing to the corresponding point on the presented Binary Entropy graph.
    \end{itemize}
\end{itemize}

\begin{itemize}
    \item \textbf{FR-2} From the web, it must be possible to run the Conditional Entropy calculator for calculating the conditional entropy of given input values.
    \begin{itemize}
        \item \textbf{FR-2.1} The user must be able to enter values into the presented input fields which are positioned in the columns that are given the name ``Class 1" and ``Class 2" by the respective column headers.
        \item \textbf{FR-2.2} The user must be able to add categories by clicking on the button labeled ``+''.
        \item \textbf{FR-2.3} The user must be able to remove the row that represents the category that was last added by clicking on the button labeled ``-''.
        \item \textbf{FR-2.4} The user must be able to initialize the calculation of the conditional entropy by clicking on the button labeled ``Calculate Conditional Entropy''.
        \item \textbf{FR-2.5} The application must, given valid input values, correctly calculate each category's ratio, entropy, and the feature's conditional entropy and display those values on the table.
    \end{itemize}
\end{itemize}

\subsection{Non-functional requirements}
\begin{itemize}
    \item \textbf{NFR-1} The user interface must be simple and intuitive.
    \item \textbf{NFR-2} The application must be responsive to different screen sizes.
    \item \textbf{NFR-3} For the Entropy calculator and Conditional Entropy calculator, the application must recognize any positive integer value as valid input.
    \begin{itemize}
        \item \textbf{NFR-3.1} The user must be warned through appearing alerts if any of the user-made inputs is invalid.
    \end{itemize}
    \item \textbf{NFR-4} For the Entropy calculator, if more than 2 classes are used, the user must be informed through an appearing alert about the fact that the calculated results will not be displayed on the Binary Entropy graph.
\end{itemize}

\section{Requirements specification}
\subsection{Use case diagram}

\subsection{Use cases}
% Use case 1
\begin{table}[p]
	\centering
	\begin{tabularx}{\linewidth}{ p{0.21\columnwidth} p{0.71\columnwidth} }
		\toprule
		\textbf{UC-1}    & \textbf{Run Entropy calculator}\\
		\toprule
		\textbf{Version}              & 1.0    \\
		\textbf{Author}                & Daniel Drefs Fernandes \\
		\textbf{Associated requirements} & FR-1, FR-1.1, FR-1.2, FR-1.3, FR-1.4, FR-1.5, FR-1.6 \\
		\textbf{Description}          & The user runs the Entropy calculator with the desired input values and receives the results in a visual format on the website. \\
		\textbf{Precondition}         & The input values introduced by the user are valid. \\
		\textbf{Actions}             &
		\begin{enumerate}
			\def\labelenumi{\arabic{enumi}.}
			\tightlist
			\item The user opens the application.
            \begin{enumerate}
                \item The user adds one or multiple class by clicking the button with the label ``+''.
            \end{enumerate}
			\item The user fills the input fields with the desired values and clicks on the button with the label ``Calculate Entropy''.
            \item The application calculates each class's p-value and the feature's entropy and displays the results on the corresponding table.
            \begin{enumerate}
                \item If the user has not added any classes, the application will show a visualization of the calculated results in SVG format on the Binary Entropy graph.
            \end{enumerate}
		\end{enumerate}\\
		\textbf{Postcondition}        & The results are displayed on the Entropy table. \\
		\textbf{Exceptions}          & If the user has introduced invalid values, the application will display an alert and inform the user to only use positive integer values. \\
		\textbf{Importance}          & High \\
		\bottomrule
	\end{tabularx}
	\caption{UC-1 Run Entropy calculator.}
\end{table}

\subsection{Use cases}
% Use case 1
\begin{table}[p]
	\centering
	\begin{tabularx}{\linewidth}{ p{0.21\columnwidth} p{0.71\columnwidth} }
		\toprule
		\textbf{UC-2}    & \textbf{Run Conditional Entropy calculator}\\
		\toprule
		\textbf{Version}              & 1.0    \\
		\textbf{Author}                & Daniel Drefs Fernandes \\
		\textbf{Associated requirements} & FR-2, FR-2.1, FR-2.2, FR-2.3, FR-2.4, FR-2.5 \\
		\textbf{Description}          & The user runs the Conditional Entropy calculator with the desired input values and receives the results in a visual format on the website. \\
		\textbf{Precondition}         & The input values introduced by the user are valid. \\
		\textbf{Actions}             &
		\begin{enumerate}
			\def\labelenumi{\arabic{enumi}.}
			\tightlist
			\item The user opens the application.
            \begin{enumerate}
                \item The user adds one or multiple categories by clicking the button with the label ``+''.
            \end{enumerate}
			\item The user fills the input fields with the desired values and clicks on the button with the label ``Calculate Conditional Entropy''.
            \item The application calculates each category's ratio, entropy, and the feature's conditional entropy and displays the results on the table.
		\end{enumerate}\\
		\textbf{Postcondition}        & The results are displayed on the table. \\
		\textbf{Exceptions}          & If the user has introduced invalid values, the application will display an alert and inform the user to only use positive integer values. \\
		\textbf{Importance}          & High \\
		\bottomrule
	\end{tabularx}
	\caption{UC-2 Run Conditional Entropy calculator.}
\end{table}
