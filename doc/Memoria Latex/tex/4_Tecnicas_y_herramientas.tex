\capitulo{4}{Técnicas y herramientas}

Esta parte de la memoria tiene como objetivo presentar las técnicas metodológicas y las herramientas de desarrollo que se han utilizado para llevar a cabo el proyecto. Si se han estudiado diferentes alternativas de metodologías, herramientas, bibliotecas se puede hacer un resumen de los aspectos más destacados de cada alternativa, incluyendo comparativas entre las distintas opciones y una justificación de las elecciones realizadas. 
No se pretende que este apartado se convierta en un capítulo de un libro dedicado a cada una de las alternativas, sino comentar los aspectos más destacados de cada opción, con un repaso somero a los fundamentos esenciales y referencias bibliográficas para que el lector pueda ampliar su conocimiento sobre el tema.

\section{Bootstrap}
Bootstrap ~\cite{bootstrap5_tutorial} is a front-end framework which is known for providing useful and easy-to-use HTML and CSS templates like tables, buttons, forms and many others. It also comes with JavaScript components like modal dialogues and dropdown menus.

One of Bootstrap's key features is its grid system which lets users divide their web page's contents into rows and columns:
\imagen{bootstrap_grid_system}{The Bootstrap grid system}{.9}
Each row possesses 12 columns which the user can freely utilize to organize the contents that are to be displayed. Bootstrap also automatically puts the device on which the web page is displayed into one of six different size categories, based on the device's screen width. This, in combination with the grid system, allows the developer to make their website responsive to different screens, ranging from large desktop monitors to smartphones.

In addition to Bootstrap's ability to create responsive websites, the wide variety of CSS classes used for common HTML elements like buttons offer further simplicity to the web design aspect of creating a page. With this, maintaining a visual consistency throughout the project is made easier, too.

In this project, the newest version of Bootstrap at present, Bootstrap 5, is used. The main advantages are its usage of vanilla JavaScript for its components instead of relying on jQuery, new components for better customization and simplified CSS which reduces file size and loading times for the created pages.

\section{D3.js}
"D3" ~\cite{d3js_what_is_d3} stands for "data-driven documents" and perfectly describes the free, open-source JavaScript library. "Documents" refers to the Document Object Model (DOM). D3.js lets the user bind data to its elements. The library works like a toolbox that uses a variety of discrete modules which, e.g., allow selection and transition operations. It binds these modules together so all the necessary tools are at hand, ready to be applied.

D3.js does not invent new data presentation formats, instead, it makes use of web standards like SVG to display contents. Incorporating these standards, the library also allows the use of external stylesheets which can be employed to change the graphics' visual representations.

A major feature of D3.js is its ability to dynamically change the displayed contents. Whether that change is triggered by user interactions or a change in underlying data, the library's data join concept allows separate operations for entering, updating and exiting existing DOM elements based on a given set of data. Besides filtering and sorting, it lets you control what happens to your contents in many ways when changes happen and update your website accordingly. 