\capitulo{2}{Project objectives}

The primary objective of this project has been to create an interactive and informative web application focused on educating users about the decision tree algorithm ID3.
To achieve this goal, the user would first have to be explained the concepts of entropy and conditional entropy which are prerequisites for understanding decision trees and their creation. 

That is why the first part of the application serves as a tool with which users can learn about entropy and how it is calculated. Alongside cards that hold general information about entropy, the user is provided with a calculator that lets them calculate the entropy for self-chosen values and a visual representation of the binary entropy function.

The conditional entropy calculator works in the same way. Users can input different values that represent the number of instances for an example category of an example feature and calculate the resulting conditional entropy.

The main focus of the decision tree simulator is the dynamic decision tree creation that makes use of Scalable Vector Graphics (SVG) and whose values are based on the underlying dataset. A data table shows the entirety of the currently loaded dataset. The user can choose between utilizing one of the several example datasets provided by the application and their own datasets that have to be in a CSV file format. The step-by-step visualization of the algorithm provides the user with another table that displays relevant values at each step, e.g., a feature's information gain. In addition to that, the application presents visual cues that signal the current state of the algorithm at each step.