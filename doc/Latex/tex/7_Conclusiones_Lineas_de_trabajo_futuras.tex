\capitulo{7}{Conclusions and future lines of work}

\section{Conclusions}
One of the hurdles along the way was to acquire knowledge about the technologies and tools that were necessary to carry out this project in a satisfactory way. This was successfully achieved for technologies like JavaScript, CSS, Bootstrap, D3, and others that were relevant for web development.
With that, the main objectives were completed in a way that realizes the initial vision of the project. A web application was built that shows the concept of decision trees, the behavior of the ID3 algorithm, and surrounding concepts like entropy.
It is a tool that can help students make this topic more tangible and simple to understand during their studies.

\section{Future lines of work}
\subsection{Resolve shortcomings}
Firstly, a future expansion of this project could resolve all the shortcomings. Mentioned in \ref{decision_two_features} and \ref{limitations_small_app}, these include the overlapping of branch paths with nodes, CSV file limitations for users due to the size of the application, and the lack of the feature that allows for interactive data. 

\subsection{Questionnaires}
Much like in the research article~\cite{https://doi.org/10.1002/cae.22036} about the ``Seshat Tool'' that was mentioned in the introduction, a similar study could be made about the effectiveness of this decision tree simulator. Two groups of students could be asked to take the same questionnaire about the topics of entropy, decision trees and their creation using the ID3 algorithm. One group would be prepared only through theoretical input in form of a lecture, the other would additionally be given access to this application. The results would be analyzed to see if there was a difference in test scores between the two groups.

\subsection{C4.5 algorithm}
The C4.5 algorithm~\cite{c4_5_wiki} is an algorithm that extends ID3 by being able to handle both continuous and discrete attributes, handle missing attribute values, and prune trees to avoid overfitting. A future work could expand upon this application and implement a simulator for the C4.5 algorithm.

Furthermore, a scatter plot could be used to present how a tree with numerical values works. In this way, the understanding of the user of that type of decision trees could be strengthened while additionally, the concept of decision boundaries could be explained in an interactive way.

\subsection{Regression trees}
A future addition to this application could implement a simulator that teaches the concept and creation of regression trees. E.g., the Classification And Regression Tree (CART) algorithm~\cite{cart}, which, as the name suggests, is an algorithm that is applicable for classification and regression problems, could be taught interactively.