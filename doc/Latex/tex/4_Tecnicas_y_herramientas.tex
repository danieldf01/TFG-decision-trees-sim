\capitulo{4}{Techniques and tools}

In this section, all the development tools that have been used to carry out the project are presented. This ranges from frameworks like Bootstrap to version control tools like GitHub. Besides a short introduction, their most notable functions and benefits to the development of the project are documented.

\section{Bootstrap}
Bootstrap~\cite{bootstrap5_tutorial} is a front-end framework which is known for providing useful and easy-to-use HTML and CSS templates like tables, buttons, forms and many others. It also comes with JavaScript components like modal dialogues and dropdown menus.

One of the key features of Bootstrap is its grid system which lets users divide the contents of their web page into rows and columns. Figure \ref{fig:bootstrap_grid_system} shows how the grid system is structured.
\imagen{bootstrap_grid_system}{The Bootstrap grid system \cite{bootstrap_5_grid_w3s}}{1}
Each row possesses 12 columns which the user can freely utilize to organize the contents that are to be displayed. Bootstrap also automatically puts the device on which the web page is displayed into one of six different size categories, based on the screen width of the device. This, in combination with the grid system, allows the developer to make their website responsive to different screens, ranging from large desktop monitors to smartphones.

In addition to the ability of Bootstrap to create responsive websites, the wide variety of CSS classes used for common HTML elements, like buttons, offer further simplicity to the web design aspect of creating a page. With this, maintaining a visual consistency throughout the project is also made easier.

In this project, Bootstrap 5 is used. One of the Bootstrap features used is shown in figures \ref{fig:bootstrap_proof_code} and \ref{fig:bootstrap_proof_alert}, which displays a custom alert to the user, making use of the grid system of Bootstrap with the ``row'' and ``col'' CSS classes and the ``alert'' class for the alert.
\imagen{bootstrap_proof_code}{Usage of Bootstrap}{1}
\imagen{bootstrap_proof_alert}{Bootstrap Alert}{1}

\section{D3.js} \label{d3js}
D3~\cite{d3js_what_is_d3} stands for ``data-driven documents'' and perfectly describes the free, open-source JavaScript library. ``Documents'' refers to the Document Object Model (DOM). D3 lets the user bind data to its elements. The library works like a toolbox that uses a variety of discrete modules which, e.g., allow selection and transition operations. It binds these modules together so all the necessary tools are at hand, ready to be applied.

D3 makes use of web standards like SVG to display contents. Incorporating these standards, the library also allows the use of external stylesheets which can be employed to change the visual representations of the graphics.

A major feature of D3 is its ability to dynamically change the displayed contents. Whether that change is triggered by user interactions or a change in underlying data, the ``data join'' concept of the library allows separate operations for entering, updating and exiting existing DOM elements based on a given set of data. Besides filtering and sorting, it lets the user control what happens to their contents in many ways when changes happen and update their website accordingly.

Figures \ref{fig:d3js_proof_graph}, \ref{fig:d3js_proof_drawpoint}, and \ref{fig:d3js_proof_entropygraph} show the usage of D3 in this project. It was used to create a coordinate system with a graph displaying the binary entropy function. It is also used to dynamically add a point and line to the graph, corresponding to the binary entropy value that was calculated for the input values of the user:
\imagen{d3js_proof_graph}{Usage of D3: Defining the chart area}{1}
\imagen{d3js_proof_drawpoint}{Usage of D3: drawing the point and line}{.6}
\imagen{d3js_proof_entropygraph}{D3 graph}{.5}
\pagebreak

\section{PyCharm} \label{pycharm}
PyCharm~\cite{pycharm} is the Integrated Development Environment (IDE) that was used during the first two and part of the third sprint. It is an IDE developed by JetBrains which is specifically designed for Python development. With its support for web development frameworks like Flask and features like code completion for HTML, CSS and JavaScript, it served useful in allowing a quick start into the development of this project.
The built-in live preview for HTML files and its interactive debugging feature also made working on the prototypes much simpler. Figure \ref{fig:pycharm_proof_livepreview} shows the usage of that feature.
\imagen{pycharm_proof_livepreview}{Live preview of PyCharm}{1}
Its version control integration of systems like Git allowed an easier experience of making local changes remotely available.

However, one of the issues during the third sprint was to start using Bootstrap to make the layout of the website responsive. Due to set-up problems with the IDE, PyCharm was not able to provide auto-completion for the framework. To make use of that and other features, a different program overtook the task of developing the application.

\section{Visual Studio Code}
Visual Studio Code~\cite{vscode} is an open-source code editor created by Microsoft. While it also provides all the benefits mentioned in section \ref{pycharm}, it comes with a handful of other advantages of which the most significant one would be the vast variety of available extensions.
These include Bootstrap IntelliSense which enables CSS class auto-completion, Live Server which launches a local server with a live reload feature, and GitLens which, e.g., allows the user to see inline information about when and in which commit the current line of code was last changed. Figure~\ref{fig:vscode_proof_gitlens} shows this feature.
\imagen{vscode_proof_gitlens}{GitLens extension}{1}
These extensions allow the user to add features based on what they need. By only including essential coding features upon first installation, the editor take up less space than IDEs like PyCharm.
In addition, its lightweight design that is optimized for performance leaves behind a smaller memory footprint.

\section{GitHub}
GitHub~\cite{github} is a platform that is mainly used for version control on software development projects. Not only does it provide a location for users to store their files, make and track changes, it also makes it possible to share repositories with other developers and therefore presents an easy way to collaborate on projects.

Project management is made easier with GitHub, too. Providing tools for code review, milestone and issue tracking, it gives participants of the project insight on the current progress. Communication through comments on specific issues, commits, pull requests or code reviews is possible, too, allowing for organized feedback.

In this project, GitHub has been used to store all relevant files and update the development in 
the repository\footnote{GitHub: \url{https://github.com/danieldf01/TFG-decision-trees-sim/}}. Issues were used to set up defined tasks and track progress. In addition to that, a GitHub Pages repository\footnote{GitHub Pages repository: \url{https://danieldf01.github.io/}} was used to deploy prototypes for different functionalities of the web application. GitHub Pages is a site hosting service that allows the user to host a website directly from a GitHub repository. It simplified the sprint reviews with the tutors as the deployed web application could simply be opened on any browser without having to launch it from an IDE or a code editor.

All these functions have been in the present work to ensure an organized and transparent development process.

\section{Zube}
Zube~\cite{zube} is a project management platform for software development teams that comes with tools that help to plan and manage projects efficiently. Allowing a direct link to GitHub repositories, issues are synced so that when a change is made to those linked issues on GitHub, that information is transferred to Zube, and vice versa.

Tracking progress is made easy with the inclusion of sprints and allowing the user to add issues to a sprint board which can be used for project management methodologies like Scrum. Charts like Burndown and Burnup show the progress of each sprint over the time set for the sprint. The Burndown charts were used to analyze the temporal planning throughout the project in the corresponding section in the annexes.

Figure \ref{fig:sprint4_screenshot} shows the Sprint Board which, among other tools, was used to plan out each sprint thoroughly and document each step in regards to the created issues.
\pagebreak
\imagen{sprint4_screenshot}{Sprint Board}{1}
\pagebreak

\section{Codacy} \label{codacy}
Codacy~\cite{codacy} is a code analysis tool that helps software developers improve their code quality. Allowing a link to GitHub, it reviews commits in real-time and checks the quality of the code based on code security, code duplication, coding style and other general issues.
The coding standards that are initially introduced by Codacy can be customized to fit the quality standards of the developres by making it possible to, e.g., ignore certain issues that are irrelevant to the project, introduce other security guidelines and enforce specific coding styles of their own.

As seen in figure \ref{fig:codacy_screenshot}, which shows the timeline of existing issues within the code in the remote repository, Codacy has been used to continuously identify and tackle all sorts of problems in the code that arose during the development of the web application.
\imagen{codacy_screenshot}{Codacy quality evolution}{1}